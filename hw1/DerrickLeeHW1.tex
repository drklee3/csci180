\documentclass[12pt]{exam}
\usepackage[colorlinks]{hyperref}
\usepackage{enumitem}
\usepackage{float}
\usepackage{listings}
\usepackage{multicol}
\usepackage{xcolor}

\lstset{
    showstringspaces=false,
    breaklines=true,
    basicstyle=\footnotesize,
    commentstyle=\color{gray}
}

\begin{document}
\noindent
Derrick Lee\\
Homework 1\\
CSCI 180\\
\today\\

\section{}

This function is vulnerable to buffer overflow. Since there is a char buffer of
30 bytes and \lstinline{gets(str)} is used without checking for overflow.  This
results in possibly attempting to store more data than the buffer. The stack
frame is as follows:

\begin{table}[h]
\centering
\begin{tabular}{| l |}
\hline
str{[}30{]} \\ \hline
SFP         \\ \hline
ret         \\ \hline
\end{tabular}
\end{table}

If a user has an input that is over 30 bytes, it will overflow the buffer and
consequently overwrite the previous frame pointer along with the return address.
This will result in various possible ways if the return address is modified:

\begin{enumerate}
    \item The return address points to a valid instruction elsewhere in the
          memory, potentially in the given buffer, thus allowing external code
          execution.
    \item The return address points to a protected or invalid location causing
          a segmentation fault.
\end{enumerate}

\section{}

The main issue is the lack of input validation associated with the memory
allocation resulting in an attack with integer overflow.  With the given
\lstinline{len} from the network, any number can be provided.  The numbers that
would cause issues are between \lstinline{UINT_MAX - 9} and \lstinline{UINT_MAX}
(\lstinline{UINT_MAX} being 4294967295 on a 32 bit system or
18446744073709551615 on a 64 bit system).  

Assuming \lstinline{UINT_MAX} is provided and assigned to \lstinline{len}, the
\lstinline{malloc(len + 10)} will result in an allocation of \lstinline{UINT_MAX + 10}
which is equivalent to 9.  With 9 bytes allocated to \lstinline{buf} and
\lstinline{info} is read into buf with a length of \lstinline{UINT_MAX}, the
data copied is larger than the buffer size causing a buffer overflow.

\section{}

\begin{enumerate}[listparindent=1.25em]
    \item \textbf{Off-by-One error}

          The for loop iterates from 0 to <= n which would result in an index
          value that is one past the end of the input shoplist array.

    \item \textbf{Buffer Overflow}

          When using \lstinline{snprintf()}, the input food name is copied to
          the next position in the \lstinline{expensive} buffer.  Because each
          food name is 1024 bytes and the expensive buffer is also 1024 bytes,
          having at least two elements in the shoplist can produce a buffer
          overflow as the sum of the shoplist name lengths (for items priced
          over 70) can be greater than the expensive buffer size.

          Another area that could produce a buffer overflow is adding len to
          \lstinline{size_exp}.  Since len is a \lstinline{size_t} and
          \lstinline{size_exp} is an int, adding enough to \lstinline{size_exp}
          can cause an integer overflow (possibly negative) and thus allow
          snprintf write to a memory location that isn't intended. 
    
    \item \textbf{OS Command Execution}
    
          If the total is greater than 1200, a command will be executed on the
          system that contains a list of the expensive items from the input food
          names.  Since the list of expensive food names are listed without
          input sanitation, any command can be executed on the system by the
          user.  If the program is run as root, then this allows a malicious
          user to execute any command as root.

          For example, if the total is over 1200 and an expensive food item is
          given the name \lstinline{"; cat /etc/passwd"} or 
          \lstinline{"&& sudo rm -rf /"} (a \lstinline{"#"} can be added at
          the end if there are to be items after the food name to be commented
          out and thus ignored.  e.g. \lstinline{"&& cat /etc/passwd #"}).
\end{enumerate}

\end{document}