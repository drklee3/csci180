\documentclass[12pt]{exam}
\usepackage{enumitem}
\usepackage{listings}
\usepackage{xcolor}
\usepackage{multicol}

\lstset{
    showstringspaces=false,
    breaklines=true,
    basicstyle=\footnotesize,
    commentstyle=\color{gray}
}

\begin{document}
\noindent
Derrick Lee\\
CSCI 180\\
Lab Exercise 2\\
\today\\

\section{}

Modes used:
\begin{itemize}[noitemsep]
    \item wordlist (given dictionary + online password list)
    \item wordlist + rules (default, Single, KoreLogic)
    \item Markov
    \item Incremental (default)
    \item Incremental (charset with cracked passwords)
\end{itemize}

\noindent
Commands run that cracked passwords (with corresponding user passwords):
\begin{lstlisting}[language=bash]
    $ john --wordlist=dictionary.txt --format=raw-MD5 target.txt 
    # user87, user88, user92, user93, user94, user00, user98

    $ john --wordlist=dictionary.txt --format=raw-MD5 --rules target.txt 
    # user91, user89, user15, user97, user32, user14, user13, user12, user99, user02, user01

    $ john --incremental=digits --format=raw-MD5 target.txt 
    # user95, user56

    $ john --markov --max-length=20 --format=raw-MD5 target.txt 
    # user79, user84, user86, user90, user20, user96

    $ john --wordlist=password.lst --rules --format=raw-MD5 target.txt 
    # user49, user39, user19, user31

    $ john --max-length=12 --format=raw-MD5 target.txt 
    # user11, user10, user09, user36, user72

    $ john --wordlist=dictionary.txt --rules=all --format=raw-MD5 target.txt 
    # user05, user03, user28

    $ john --incremental=charset --format=raw-MD5 target.txt 
    # user18, user74, user47

    $ john --wordlist=10-million-password-list-top-100000.txt --format=raw-MD5 target.txt 
    # user29

    $ john --wordlist=10-million-password-list-top-100000.txt --rules --format=raw-MD5 target.txt 
    # user07, user06, user40

    $ john --wordlist=10-million-password-list-top-100000.txt --rules=KoreLogic --format=raw-MD5 target.txt 
    # user67, user66, user85, user57, user75

    $ john --wordlist=dictionary.txt --rules=KoreLogic --format=raw-MD5 target.txt 
    # user65

    $ john --wordlist=10-million-password-list-top-100000.txt --rules=Single --format=raw-MD5 target.txt 
    # user69

    $ john --wordlist=10-million-password-list-top-1000000.txt --format=raw-MD5 target.txt 
    # user46

    $ john --wordlist=10-million-password-list-top-1000000.txt --rules --format=raw-MD5 target.txt 
    # user48
\end{lstlisting}

\noindent
Commands with no cracked users or canceled due to excessive runtime:
\begin{lstlisting}[language=bash]
    $ john --wordlist=password.lst --format=raw-MD5 target.txt 
    $ john --wordlist=all.lst --rules --format=raw-MD5 target.txt 
    $ john --wordlist=passwords.lst --rules --format=raw-MD5 target.txt 
    $ john --mask=?1?1?1?1?1?1?1?1 --1=[A-Z] --format=raw-MD5 target.txt 
    $ john --mask=?1?1?1?1?1?1?1?1 --1=[A-Z] --min-length=8 --format=raw-MD5 target.txt 
    $ john --markov=200 --max-length=7 target.txt --mkv-stats=markovstats 
    $ john --markov=200 --max-length=7 --format=raw-MD5 target.txt --mkv-stats=markovstats 
    $ john --markov=10 --max-length=7 --format=raw-MD5 target.txt --mkv-stats=markovstats 
    $ john --markov=100 --max-length=7 --format=raw-MD5 target.txt --mkv-stats=markovstats 
    $ john --incremental=charset --max-length=12 --format=raw-MD5 target.txt 
    $ john --wordlist=dictionary.txt --rules=Wordlist --format=raw-MD5 target.txt 
    $ john --wordlist=dictionary.txt --rules=Extra --format=raw-MD5 target.txt 
    $ john --wordlist=dictionary.txt --rules=Wordlist --format=raw-MD5 target.txt 
    $ john --wordlist=dictionary.txt --rules=Extra --format=raw-MD5 target.txt 
    $ john --wordlist=10-million-password-list-top-100000.txt --rules --format=raw-MD5 target.txt 
    $ john --wordlist=10-million-password-list-top-1000000.txt --rules=Single --format=raw-MD5 target.txt 
    $ john --wordlist=10-million-password-list-top-1000000.txt --rules=Single --format=raw-MD5 target.txt 
    $ john --wordlist=10-million-password-list-top-1000000.txt --rules=KoreLogic --format=raw-MD5 target.txt 
    $ john --incremental=charset --format=raw-MD5 target.txt 
    $ john --incremental=charset --max-length=10 --format=raw-MD5 target.txt
    $ john --loopback --rules --format=raw-MD5 target.txt 
    $ john --loopback --rules=all --format=raw-MD5 target.txt 
    $ john --loopback --rules=all --format=raw-MD5 target.txt 
    $ john --mask=?1?1?1?1?1?1??1?1 --1=[A-Z] --min-length=8 --format=raw-MD5 target.txt 
    $ john --markov --max-run-time=10 --format=raw-MD5 target.txt 
    $ john --markov --max-run-time=50 --format=raw-MD5 target.txt 
    $ john --incremental=Digits --format=raw-MD5 target.txt 
    $ john --incremental=Digits --max-length=4 --format=raw-MD5 target.txt 
    $ john --incremental=Digits --min-length=4 --max-length=8 --format=raw-MD5 target.txt 
    $ john --incremental=Digits --min-length=8 --max-length=16 --format=raw-MD5 target.txt 
    $ john --incremental=Digits --min-length=8 --max-length=12 --format=raw-MD5 target.txt 
    $ john --incremental=Alnum --max-length=8 --format=raw-MD5 target.txt 
    $ john --incremental=Alnum --max-length=4 --format=raw-MD5 target.txt 
    $ john --incremental=Alnum --max-length=6 --format=raw-MD5 target.txt 
    $ john --incremental=ASCII --max-length=6 --format=raw-MD5 target.txt 
\end{lstlisting}

\noindent
Passwords cracked:
\begin{multicols}{2}
    \begin{lstlisting}
        user00:hashemi
        user01:8ferret
        user02:ruben6
        user03:criminal16
        user05:f00tba11
        user06:dingding1
        user07:goodday1
        user09:babigirl1
        user10:candy1992
        user11:sunset15
        user12:homedepot5
        user13:riverside!
        user14:butthead2
        user15:motorhead1
        user18:1mateo4
        user19:1medical
        user20:1memme
        user28:1nothing1
        user29:1nternet
        user31:1onelove
        user32:1orange
        user36:1susan2
        user39:1teddybear
        user40:1texasboy
        user46:1Vipers
        user47:1webstar
        user48:1westsider
        user49:1winnie
        user56:20013694
        user57:2006acura
        user65:20hopedale
        user66:20inches
        user67:20september
        user69:210592w
        user72:2123546a
        user74:212head
        user75:212sammyd
        user79:maxx13
        user84:a1234666
        user85:21norway
        user86:portinga
        user87:casper
        user88:badone
        user89:lebanon1
        user90:fildaman
        user91:tacoma1
        user92:brookstone
        user93:knockers
        user94:braindamage
        user95:8661234
        user96:iamadam
        user97:smoesmoe
        user98:qwertyui
        user99:bubbles4
    \end{lstlisting}
\end{multicols}

\section{}

Using a variety of different modes listed above, my password was not cracked.
Since I use a password manager (previously KeePass and now Bitwarden), many of
my passwords are randomly generated, and thus difficult to crack.  Depending on
the website, they range from short (around 16 characters) to longer ones (up to
64 characters).

An example password would be as follows (\textbf{NOT} a password in use, this is newly
generated just for this assignment):

\begin{lstlisting}
    M&wncHtDTTWfp^merr^KPEd8%m*N9ef3
\end{lstlisting}

The more effective methods of password cracking from Q1 were dictionary based
attacks which would not be able to crack a randomly generated password as shown
above (assuming it’s not a reused password that could potentially end up on a
password list online).

Using randomly generated passwords provides very good protection against various
attacks.  For example the one above has a character set of uppercase and
lowercase characters, numbers, and special characters (\lstinline{!@#$%^&*}) --
a total of 70 possible characters.

With a minimum of 16 characters, the number of password guesses required in
in order to correctly find a given random password is as follows:

$$70^{16} = 332329305696010000000000000000$$

For even longer passwords up to 64 characters long:

$$70^{64} \approx 1.219 \cdot 10^{118}$$

Without having parts of the password as words or phrases, the only option to
crack these passwords would be an incremental brute force attack which would
require an immense amount of time and computational power.  This is, again,
assuming that these passwords were not leaked elsewhere in a plaintext database.
If a password is reused on a website or service that stored passwords in
plaintext and that password were to be part of a data breach, it would be part
of a dictionary attack to easily retrieve the same password that is properly
hashed.

\vspace*{.51in}

While a lot of of my passwords are randomly generated, there are still a number
of them that are memorized, such as the password to unlock my password manager.
These passwords are quite lengthy with a minimum of 20 characters.  While not
as secure, they're still also difficult to crack.  Some of my older passwords
are not very secure, found in several data breaches and along with some
associated older recently accounts hijacked (though they're not in use and the
passwords were easily changed to a random one).  


\end{document}