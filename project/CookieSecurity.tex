\documentclass[conference,12pt]{IEEEtran}
\IEEEoverridecommandlockouts
% The preceding line is only needed to identify funding in the first footnote. If that is unneeded, please comment it out.
\usepackage{cite}
\usepackage{amsmath,amssymb,amsfonts}
\usepackage{algorithmic}
\usepackage{listings}
\usepackage{graphicx}
\usepackage{textcomp}
\usepackage{xcolor}

\lstset{
    showstringspaces=false,
    breaklines=true,
    basicstyle=\footnotesize,
    commentstyle=\color{gray},
    frame=single,
    captionpos=b
}

\def\BibTeX{{\rm B\kern-.05em{\sc i\kern-.025em b}\kern-.08em
    T\kern-.1667em\lower.7ex\hbox{E}\kern-.125emX}}
\begin{document}

\title{Cookies and Session Vulnerabilities}

\author{\IEEEauthorblockN{Derrick Lee}
\IEEEauthorblockA{dlee3@scu.edu}
\and
\IEEEauthorblockN{Another person}
\IEEEauthorblockA{email}
}

\maketitle

\begin{abstract}
Cookies provide the fundamental backbone of modern web services.  There are a
number of security practices that are important to keep in mind to prevent
attacks relating to cookies.
\end{abstract}

\section{What Are Cookies}

Cookies are part of the basis of what we consider the internet.  From websites
keeping you logged in to ad services tracking you across them, cookies allow
states to be preserved across an otherwise stateless HTTP protocol.  This is
because the internet was built stateless, so each HTTP request makes a new
connection and doesn’t remember previous ones.  

With the advancement of more complex websites and services requiring the need to
maintain a state, cookies are used to send information in between the client and
server.  Utilizing this additional information that can be stored in the browser
and sent back to the server in the next request opens up opportunities to add
extra features that wouldn’t have been possible without cookies.  Such examples
are keeping users logged in, saving personalized settings, and recording user
behavior \cite{b1}.

Cookies take the form as a key and value.  Various different kinds of data can
be stored as the value, encoded as a string.  Cookies are set by the server via
the \lstinline{Set-Cookie} HTTP response header and are sent back from the
browser for subsequent requests, also as a cookie header.

\begin{lstlisting}[caption={Server response to set a cookie}]
HTTP/2.0 200 OK
Content-type: text/html
Set-Cookie: a_cookie=my_cookie_value
Set-Cookie: another_cookie=yum

[page content]
\end{lstlisting}

\begin{lstlisting}[caption={Client request including cookie}]
GET /sample_page.html HTTP/2.0
Host: www.example.org
Cookie: a_cookie=my_cookie_value; another_cookie=yum
\end{lstlisting}

\section{How Are Cookies Used?}

In what ways do cookies benefit, make web browsing smoother, or more convenient
for the user? At least for the case of preserving user logins, personalizing the
browser, and providing added security, cookies are currently the
easiest-to-implement solution that provides all these services at once.

\subsection{Session Handling}

One of the biggest use cases, and one of the most sensitive, is using cookies to
preserve the state for user logins.  This is done with a session cookie which
saves an identifier on the user’s browser and has the corresponding session data
on the server.  When requests are made from a user’s browser, the session cookie
is also sent which allows the server to identify the user and match them to the
correct session data.

While this is convenient for the user, this function can pose some additional
security vulnerabilities, and a more detailed explanation on how cookies handle
the protection of user information can be found in the following section about
"Cookie Vulnerability.”

\subsection{Auto-Form Completion}

Similar to session handling, cookies can be used to save user input of personal,
yet repetitive information like full names, email addresses, and date of births.
This information is packaged into a cookie and sent to the web browser, which
will help users automatically fill out this data the next time the same
information is requested. While it might seem like an invasion of privacy, users
only have to fill out cumbersome forms-- like shipping or billing information,
medical or insurance forms, and the FAFSA -- once. 

\subsection{Personalization}

Various user settings can be saved in cookies as well.  While a lot of settings
are saved on the server or in other more modern and efficient storage APIs like
\lstinline{localStorage} and \lstinline{sessionStorage}, cookies can be used to
store things like a user’s preferred language, location, or theme.  For example,
if a website has an optional dark mode, a cookie could be used to save such an
option and provide the user with the website in a dark theme.  In this specific
example, the same functionality can be achieved in modern websites when
rendering is done on the client \cite{b2}.

\section{Cookie Security}

While cookies allowed a more featureful internet, it also increased the
additional attack surface.  Security was not in the original feature set,
containing neither confidentiality protection nor integrity for the data sent
between the client and server.  With these vulnerabilities, there are several
ways to prevent attacks relating to cookies. 

\section{Cookie Vulnerabilities and Server Mitigations}

Many of the security vulnerabilities related to cookies can be mitigated by
properly configuring and implementing systems that interact with cookies on the
server side.  Having a secure system relies on the server to be aware of the
following concepts and take action to prevent such attacks.  A system’s
resilience is only as strong as its weakest link, so is it important to consider
all the different security practices.  Failure to do so could result in
unauthorized access to user accounts, which depending on the service could mean
breaches in private data, administrative privileges, and possibly large sums of
money.  

\subsection{Confidentiality}

Cookies, like regular HTTP requests, are sent over an unencrypted channel where
anyone can view the content.  While it is not good practice to store sensitive
data in cookies, it still has critical information like a session identifier.

To prevent people from snooping on cookies they’re not intended to view, the
secure attribute on the cookie can be set to limit the cookie to requests that
are only transmitted over HTTPS with TLS (Transport Layer Security).  This
ensures that the cookies are only sent when they’re used on an encrypted site
and hides the content from malicious users.  However, this does not provide
integrity of cookies — attackers can modify cookies from an insecure channel.
The secure flag should be used in combination with other safety mechanisms like
HSTS, described in detail in the next section.

\begin{lstlisting}[caption={An example of server response header to set a secure cookie}]
HTTP/2.0 200 OK
Content-type: text/html
Set-Cookie: my_secure_cookie=no_snooping_here; Secure

[page content]
\end{lstlisting}

Thus, communicating over encrypted HTTPS connections prevents the "middle men”
to view cookies and associated data, effective against network MITM attacks
\cite{b3}.

\subsection{Integrity and HSTS}

Defined as keeping data from being tampered with or modified, \textbf{integrity}
is a very important aspect of security. \textbf{HTTP Strict Transport Security (HSTS)}
mitigates some of the associated issues related to MITM, cookie sniffing, and
SSL stripping \cite{b4}.

There is one aspect of HSTS that is important to keep in mind — it uses a Trust
on First Use model \cite{b5}. If the first connection to the server is already
compromised, HSTS may not prevent any attacks relating to using unencrypted
channels as an attack vector.

HSTS ensures all connections after the initial HSTS header are made over HTTPS
until it expires.  The recommended duration for such a policy is over 120 days,
and ideally 1 year \cite{b6}.  This means that no unencrypted connections are
allowed to be made for the entire duration, and thus the attack window is very
small.

Another security aspect of HSTS is subdomains, the additional part of main
domain names that are useful for navigating and organizing different parts of a
site.  An example of a subdomain for example.com is subdomain.example.com.  The
"includeSubDomains” directive, should be set so that the HSTS policy will also
cover subdomains.  Otherwise if it isn’t set, subdomains will be accessible over
HTTP, opening up the attack surface.

\begin{lstlisting}[caption={Example of the HTTP Strict-Transport-Security response header}]
Strict-Transport-Security: max-age=31536000; includeSubDomains
\end{lstlisting}

The above example enforces HTTPS on all current and future subdomains for 1 year.

\subsection{Session Hijacking and XSS}

Session hijacking attacks are very harmful because they not only give attackers
access to private data, but also bypass authentication to use a victim’s
session.  For example, hijacking cookies could allow full access to a victim’s
Facebook account, email, or even bank accounts. Not only will the attacker be
able to view your private details associated with such accounts, but they can
also do anything as victim, such as sending messages, emails, money, etc.

One example of session hijacking is through \textbf{Cross-Site Scripting (XSS)}
attacks. When malicious scripts are injected into websites, possibly by
exploited 3rd party scripts or unsanitized inputs, an attacker’s script (written
in JavaScript) can read and receive a victim’s cookies.

With a malicious script, it is extremely easy to both read and modify cookies.

\begin{lstlisting}[caption={Example of reading and modifying cookies}]
<script>
document.cookie = "a_cookie=its_modified_now"; 
document.cookie = "another_cookie=attacc"; 

console.log(document.cookie); 
// logs "a_cookie=its_modified_now; another_cookie=attacc"
</script>
\end{lstlisting}

From the example above, it can be seen that reading and changing cookies is
trivial, and can be extended to send the victim’s cookies to an attacker which
can be used to hijack the victim's session.

In order to prevent XSS attacks, the HttpOnly cookie flag provides very good
protection. Since XSS attacks in most, if not all, use injected JavaScript code,
the HttpOnly is effective as it prevents the access to cookies via JavaScript.
This means attackers who inject malicious scripts cannot access cookies, and
thus user data is protected.  

\begin{lstlisting}[caption={An example of server response header to set a Secure and HttpOnly cookie}]
Set-Cookie: my_extra_secure_cookie=no_snooping_here; Secure; HttpOnly
\end{lstlisting}

Using HttpOnly cookies is the same as Secure cookies, just an addition to the
Set-Cookie response header.

\subsection{Cross-Site Request Forgery (CSRF)}

Since cookies, including ones that include session data, are sent with HTTP
requests automatically, it makes cross-site request forgery (CSRF) attacks
easier. In short, CSRF impersonates a user and sends requests as the victim with
possibly malicious actions.  These actions include inserting a url that performs
an action like sending money into an image element on a malicious website.
Doing this would cause the victim’s browser send a request for the image url,
which in reality is a url set by the attacker, that would receive the victim’s
session cookies and thus do some requests as the victim \cite{b7}.

Some examples of CSRF attacks are shown below \cite{b8}.

\begin{lstlisting}[caption={CSRF implementation with a GET request}]
<img src="https://bank.example.com/withdraw?account=bob&amount=1000000&for=mallory">
\end{lstlisting}

\begin{lstlisting}[caption={CSRF implementation with a POST request and invisible form}]
<form action="https://bank.example.com/withdraw" method="POST">
  <input type="hidden" name="account" value="bob">
  <input type="hidden" name="amount" value="1000000">
  <input type="hidden" name="for" value="mallory">
</form>
<script>window.addEventListener('DOMContentLoaded', (e) => { document.querySelector('form').submit(); }</script>
\end{lstlisting}

There are a few ways to mitigate CSRF, including idempotent GET endpoints, CSRF
tokens, SameSite attribute, and HTTP headers.  

One of the most important aspects to prevent such an attack is maintaining
idempotent GET endpoints.  In listing 1 above, the image element performs a GET
request with the malicious parameters, withdrawing money from a victim and
sending it to the attacker.  In this case, it is obvious that the request does
modifications to the victim’s account, which should not be done.
\textbf{Idempotent GET} endpoints are supposed to be "safe,” meaning that GET
only retrieves data, does not modify anything, and repeated calls do not change
the outcomes \cite{b9}.

Another way to prevent CSRF is to use CSRF tokens.  A \textbf{CSRF token} is a
randomly generated value and inserted into the HTTP request header or as a form
parameter.  This is sent with the request and is required to be correct so that
the request is accepted.  This prevents malicious requests like the second
example above, as the attacker cannot guess the token and cannot get the token
from the user’s browser \cite{b10}.

In addition to CSRF tokens, cookies can have the \lstinline{SameSite} attribute
set.  This only allows cookies to be sent to websites that the user is currently
on as a first party cookie.  For example, if cookies made by example-bank.com
have the \lstinline{SameSite} attribute, the cookie will only be sent when a
user is on example-bank.com and will not be sent if example.com sends a request
to example-bank.com.  More information about first and third party cookies are
described in the next section.

\begin{lstlisting}[caption={CSRF token with a GET request}]
https://bank.example.com/withdraw?account=bob&amount=1000000&for=mallory&token=1990AAAFF39F34E0D36C1D4380657AD5
\end{lstlisting}

With the CSRF token sent with the request, the attacker will not be able to
perform the same attack above with the HTML image tag.  Similarly, the token can
be embedded and sent as a form field to prevent attacks with POST requests.

Finally, certain HTTP headers can further prevent CSRF attacks. The
\lstinline{referer} header tells the server the previous page or the page that
sent a request \cite{b11}.  This means the server can check which page sent the
request, and if the request is sent from a 3rd party site, it can be rejected.
If a user visits malicious.example.com which sends a request to
bank.example.com, the request to bank.example.com will have the Referer HTTP
header set to malicious.example.com.  In this case, bank.example.com will be
able to check the header and handle the illegitimate request properly.

\section{Tracking and Privacy}

Cookies may seem to provide a lot of benefits to website users.  However, they
can be used -- at the cost of the user’s privacy -- to track users for analytics
or advertising purposes. Thus, awareness of third party cookies and their
possible risks are essential to maintaining user privacy. 

\subsection{Third party cookies}

When visiting a website, a third party cookie can be set by a different website
from sources like embedded JavaScript. For example, if you are visiting
example.com and a cookie is set for example-ads.com, it is considered a third
party cookie.  First party cookies, on the other hand, are set by the website
you are visiting.  An example would be example.com setting cookies for
example.com.  This means any website making requests to example-ads.com would
use the same third party cookie, and thus show your browsing habits to
example-ads.com.

Analytics services can track users on websites that aren’t directly associated
with them, and are extremely widespread on the internet.  For example, Google
Analytics is used on a large amount of websites, currently used in 67\% of the
top 1 million websites \cite{b12}.

Facebook is also known for tracking users, even if you’re not using Facebook.
If you’ve seen a Facebook like or button on a website, external resources are
fetched from Facebook, and thus passing data to the 3rd party.  This also could
happen if you don’t see any Facebook related media, such as with Facebook Pixel
which embeds invisible tracking scripts similar to Google Analytics \cite{b13}.
If you are logged into your Facebook account, your browsing habits can be
directly linked to who you are.

While tracking and analytic services extend further than cookies, it is
important to keep in mind, and one of the methods of tracking is to utilize
cookies.

In order to prevent tracking, or rather reduce the amount of tracking done,
users can disable third party cookies in browsers.  You can also install an
adblocker like uBlock Origin, many of which also block common tracking and
analytics scripts.  However, it is important to know that doing these things
does not prevent all tracking.  There are many other ways to identify users
across browsing sessions even with cookies disabled such as browser
fingerprinting, but are out of the scope of this paper \cite{b14}.

\section{Client Mitigation and Prevention}

What is a definitive way of protection from security risks? That is, to take
preventive measures of avoiding situations that compromise your security.  A
system with an absolute zero attack surface is impossible, but there are ways to
decrease risks and mitigate threats.  While a big part of security related to
cookie relies on the servers and service providers to correctly implement their
systems keeping the vulnerabilities in mind, there are a few ways users can
reduce the attack surface.

\subsection{Keeping Up To Date}

Using a modern, up to date browser is vital to having HSTS and HTTPS support.
Many sites now use HSTS and many more use HTTPS.  There are even certain
top-level domains (TLDs) such as .dev that are on the HSTS preload list, making
HTTPS required on all connections without additional configuration.

The HSTS preload list is also a valuable tool browsers use to enforce the use of
HTTPS and further reduce the attack surface of HSTS’s Trust on First Use model
by having websites hardcoded into browsers to use HTTPS only.  Most modern
browsers like Chrome, Firefox, and Safari have HSTS preload lists.  

Modern browsers also are more strict against unsecured connections, providing
the user with big warnings and possibly preventing the user from connecting when
there is an invalid security certificates (expired, wrong host, self signed, or
untrusted root certificates) or even when there is unencrypted content.

Additionally, having an up to date browser is key to having modern cryptography
so that the browser’s connection to the server uses the strongest possible
security.  Having legacy cryptography could mean the encryption being used to
transmit data over HTTPS is possibly breakable, which causes the Secure cookie
flag to be less effective.

\subsection{Using a VPN}

A Virtual Private Network (VPN) allows users to route their traffic through an
encrypted tunnel on a remote server.  Contrary to popular belief, a VPN does not
provide anonymity or extra security. While the controversies of popular VPN
providers and misbeliefs of VPNs will not be discussed in this paper, it is
important to note what a VPN can or cannot provide.  In short, a VPN will not
keep your browsing habits anonymous, won’t add extra security to unencrypted
traffic, and is most definitely not a replacement for the aforementioned
security practices.

However, one thing a VPN may provide is more privacy from your Internet Service
Provider (ISP) and more importantly other users on a public Wi-Fi network.  This
reduces the threat from MITM attackers on a local network sniffing packets or
attempting to tamper with them.  Also, this only applies to attacks on the local
network, provides little to no protection against remote attacks such as XSS and
CSRF.

\section*{Conclusion}

Cookies provide a rich world of internet services.  There may be quite a few
different ways for a malicious actor to eavesdrop or attack users and services,
but there are also many ways to prevent such attacks from happening.

Since there are so many security aspects to keep in mind when designing a system
that utilizes cookies, opportunities to attack may likely result from improper
implementations and practices from both engineers and users.  Being aware of
what attack vectors exist are key to preventing cookie attacks.

\begin{thebibliography}{00}
\bibitem{b1} \lstinline{K. LaCroix, Y. L. Loo and Y. B. Choi, "Cookies and Sessions: A Study of What They Are, How They Work and How They Can Be Stolen," 2017 International Conference on Software Security and Assurance (ICSSA), Altoona, PA, 2017, pp. 20-24.}
\bibitem{b2} \lstinline{"HTTP Cookies.” MDN Web Docs, developer.mozilla.org/en-US/docs/Web/HTTP/Cookies.}
\bibitem{b3} \lstinline{S. Sivakorn, I. Polakis and A. D. Keromytis, "The Cracked Cookie Jar: HTTP Cookie Hijacking and the Exposure of Private Information," 2016 IEEE Symposium on Security and Privacy (SP), San Jose, CA, 2016, pp. 724-742.}
\bibitem{b4} \lstinline{Xiaofeng Zheng, X. Zheng, J. Jiang, J. Liang, H. Duan, S. Chen, T. Wan, and N. Weaver, "Cookies Lack Integrity: Real-World Implications,” Proceedings of the24th USENIX Security Symposium, Washington, D.C. 2015}
\bibitem{b5} \lstinline{"Secure Your Website with HSTS.” CA Security Council, 8 Oct. 2014, casecurity.org/2014/10/08/secure-your-website-with-hsts/.}
\bibitem{b6} \lstinline{Yang, Dingjie, et al. "The Importance of a Proper HTTP Strict Transport Security Implementation on Your Web Server.” Qualys Blog, 28 Mar. 2016, blog.qualys.com/securitylabs/2016/03/28/the-importance-of-a-proper-http-strict-transport-security-implementation-on-your-web-server.}
\bibitem{b7} \lstinline{"CSRF.” MDN Web Docs, developer.mozilla.org/en-US/docs/Glossary/CSRF.}
\bibitem{b8} \lstinline{"HTTP Cookies.” MDN Web Docs, developer.mozilla.org/en-US/docs/Web/HTTP/Cookies.}
\bibitem{b9} \lstinline{Ju, et al. "Idempotent REST APIs” – REST API Tutorial, restfulapi.net/idempotent-rest-apis/.}
\bibitem{b10} \lstinline{"Cross-Site Request Forgery Prevention.” Cross-Site Request Forgery Prevention · OWASP Cheat Sheet Series, cheatsheetseries.owasp.org/cheatsheets/Cross-Site_Request_Forgery_Prevention_Cheat_Sheet.html.}
\bibitem{b11} \lstinline{"Referer.” MDN Web Docs, developer.mozilla.org/en-US/docs/Web/HTTP/Headers/Referer.}
\bibitem{b12} \lstinline{"Analytics Usage Distribution in the Top 1 Million Sites.” Analytics Technologies Web Usage Distribution, trends.builtwith.com/analytics.}
\bibitem{b13} \lstinline{"Facebook Pixel.” Facebook for Business, www.facebook.com/business/learn/facebook-ads-pixel.}
\bibitem{b14} \lstinline{"Browser Fingerprinting: What Is It and What Should You Do About It?” Pixel Privacy, pixelprivacy.com/resources/browser-fingerprinting/.}
\end{thebibliography}
\vspace{12pt}

\end{document}



