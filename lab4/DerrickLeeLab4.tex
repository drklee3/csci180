\documentclass[12pt]{exam}
\usepackage[colorlinks]{hyperref}
\usepackage{enumitem}
\usepackage{float}
\usepackage{graphicx}
\usepackage{listings}
\usepackage{multicol}
\usepackage{xcolor}

\graphicspath{ {./images/} }

\lstset{
    showstringspaces=false,
    breaklines=true,
    basicstyle=\footnotesize,
    commentstyle=\color{gray}
}

\begin{document}
\noindent
Derrick Lee\\
Lab 4\\
CSCI 180\\
\today\\

\section{Part A: p1.pcap}

\subsection{}

The IP address of the network assigning IP addresses via DHCP is 192.168.56.1

\subsection{}

\begin{lstlisting}
08:00:27:8F:4C:61 192.168.56.9
08:00:27:76:1F:7C 192.168.56.2
08:00:27:0C:66:53 192.168.56.3
\end{lstlisting}

\subsection{}

No, 192.168.56.9 (108:00:27:8F:4C:61) did not receive their IP address from
DHCP.\\
192.168.56.2 and 192.168.56.3 had IP's assigned via DHCP as they sent DHCP
requests from 0.0.0.0 to the broadcast address

\subsection{}

\begin{enumerate}[label=\alph*)]
    \item These SYN packets were sent to seemingly random ports.  This could be
    to check the target for ports that are open and listening for new
    connections.
    \item Between frames 120 and 2121, port 22 (used for SSH) responded with a
    SYN, ACK. This gave information that the server is listening on port 22.
    \item New information given between frames 2241 and 4250 is that port 80
    (used for HTTP) is also open.
    \item The packet that was returned was a SYN-ACK, which tells the client
    that the server is listening on the port and is accepting connections,
    attempting to process a TCP three way handshake.  Other ports do not respond
    with a SYN-ACK as they are not accepting connections and closed to external
    machines.
\end{enumerate}

\subsection{}

\begin{enumerate}[label=\alph*)]
    \item ARP is used to translate IP addresses to MAC addresses.
    \item 08:00:27:8F:4C:61 (192.168.56.9) is asking for the MAC address of
    192.168.56.2 and 192.168.56.3.  08:00:27:76:1F:7C (192.168.56.2) and
    08:00:27:0C:66:53 (192.168.56.3) respond with their corresponding MAC
    addresses.
\end{enumerate}

\subsection{}

\begin{enumerate}[label=\alph*)]
    \item ARP replies are valid without a request as a machine may need to
    notify others when their IP changes.
    \item These ARP replies are suspicious as they are not just assigning
    different IP addresses to their MAC address, but also that the IP addresses
    being assigned are already being used by other machines.
    \item 192.168.56.3 (08:00:27:0C:66:53) and 192.168.56.2 (08:00:27:76:1F:7C)
    are affected as their IP addresses are being redirected to point to the ARP
    broadcaster's MAC address 08:00:27:8F:4C:61.  These replies are being sent
    to these two affected machines so that IP to MAC translations between the
    two point to the attacker instead of one another.
\end{enumerate}

\subsection{}

\begin{enumerate}[label=\alph*)]
    \item The attacker was successful in executing a MITM attack.

    The corresponding machine MAC addresses (last octet) are listed for each IP
    address (last two octets):

    \begin{lstlisting}
    56.2 => 7C (real/client)
         => 61 (attacker)
    56.3 => 53 (real/server)
         => 61 (attacker)
    56.9 => 61 (real/attacker)
    \end{lstlisting}

    Attacker machine is shown in bold.

    \begin{enumerate}[label=\arabic*)]
        \item 56.2 (7C) tries to download a file from 56.3 (53).
        \item 56.2 (7C) -HTTP GET-> 56.3 (\textbf{61}) -TCP-> 56.3 (53)
        (forwarded GET request)
        \item 56.3 (53) -TCP-> 56.2 (\textbf{61}) -TCP-> 56.2 (7C) (forwarded
        GET response)
    \end{enumerate}

    The attacker intercepts connections between the victim 56.2 (7C) and the
    server 56.3 (53), acting as a proxy in between the two to both pass along
    request data and view (or potentially modify) the content.

    \item Document media type is listed in the HTTP response Content-Type header
    as \lstinline{application/pdf}.
    \item Since the document was forwarded to the victim, they would have been
    able to receive the requested pdf file as expected and thus possibly not
    notice that it was intercepted.
\end{enumerate}

\subsection{}

One method to retrieve the file being downloaded is to export the 2 pdf files
via \lstinline{File > Export Objects > HTTP}.  These files can't be opened
individually since it's a single pdf split into two parts, so they are combined
with the following command with \lstinline{cat}:

\begin{lstlisting}[language=bash]
    cat DoD5200_1ph\(1\).pdf DoD5200_1ph\(2\).pdf > DoD5200_1ph.pdf
\end{lstlisting}

An alternative method is to reconstruct the file from the packet responses
directly with the required data in packets 4576 and 7860.  Then select the
packet responses and exporting the bytes (reassembled TCP) directly via
Wireshark or copy it as a hex stream and convert it to a binary file with the
following command:

\begin{lstlisting}[language=bash]
    xxd -r -p input_file output_file
\end{lstlisting}

After this, we have two binary files that can be combined with the
\lstinline{cat} command same as above.

\begin{enumerate}[label=\alph*)]
    \item This document was published April 1997.
    \item The authority that issued this document is the DoD Directive 5200.1,
    "Information Security Program," December 13, 1996
    \item Page 25's title is "Atomic Energy Information"
\end{enumerate}

\section{Part B: p2.pcap}

\subsection{}

\begin{enumerate}[label=\alph*)]
    \item Attacker's MAC address is \lstinline{08:00:27:21:05:17}.
    \item Attacker's real IP address is 192.168.1.38.
    \item Attacker is trying to impersonate 192.168.1.1 which is usually the
    router.
    \item The victim's IP is 192.168.1.247 as the ARP packets are being sent to
    this machine.
\end{enumerate}

\subsection{}

The user searched "\lstinline{weather 32303}" in Google.  In packet 162, the
search query (\lstinline{q=weather+32303}, where space is encoded as a
\lstinline{+}) is shown in the HTTP request where the search query is provided
in a url parameter in the GET request.

The full GET request URI is the following:

\begin{lstlisting}
    /search?hl=en&client=ubuntu&hs=Zgi&channel=fs&q=weather+32303&oq=weather+32303&gs_l=serp.3..0l2j0i30l2j0i5i30l3j0i8j0i8i30l2.44912.47718.0.48648.15.9.1.5.5.0.118.808.7j2.9.0.les%3B..0.0...1c.1.4.serp.hEw2INguST0
\end{lstlisting}

\subsection{}

\begin{enumerate}[label=\alph*)]
    \item The victim next visits www.paypal.com
    \item The TLS response is sent to the attacker's machine (08:00:27:21:05:17)
    so the handshake is between the website and the attacker.
    \item The victim's username is \lstinline{TARGETUSER} and their password is
    \lstinline{ILOVETHEINTERNET}.  This was found in the HTTP POST
    application/x-www-form-urlencoded form data.  
\end{enumerate}

\end{document}
